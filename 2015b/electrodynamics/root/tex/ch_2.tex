\section{Электрическое поле. Напряженность \,\, электрического поля.}

	Сила $\vec{F}$, действующая на заряд $q$, всегда пропорциональна его величине, поэтому ~(\ref{1}) можно записать в виде
		$$\vec{F} = q\vec{E},$$
	где вектор $\vec{E}$ называют вектором напряженности электрического поля. Это аналог формулы $\vec{F}=m\vec{g}$. $\vec{E}$ и $\vec{g}$ являются характеристиками данной точки пространства.
		$$\dim{E}=\frac{\text{Н}}{\text{Кл}}=\frac{\text{В}}{\text{м}},$$
		$$\vec{F}=q\sum_i \frac{q_i\vec{r}_i}{r_i^3}=q\sum_i \vec{E}_i=q\vec{E}.$$
	Поле в данной точке есть суперпозиция всех полей, порождаемых всеми зарядами в системе.	
		$$\vec{E}_i=k\frac{q_i\vec{r}_i}{r_i^3}, \quad E_i = k\frac{q_i}{r_i^2}.$$
	Электрическое поле создается зарядами и действует на заряды. Заряды не действуют друг на друга и взаимодействуют посредством полей, которые создают.