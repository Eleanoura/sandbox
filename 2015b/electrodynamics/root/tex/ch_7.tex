\section{Изолированный проводник. Поле внутри и вне его}
	
	Проводник\index{Проводник} -- вещество, в~котором есть достаточно свободных зарядов.
	Возьмем изолированный проводник\index{Проводник!изолированный} (в~том смысле, что электроны\index{Электрон} его не~покидают) и~поместим его во~внешнее поле\index{Поле!внешнее}. Поскольку заряды свободны\index{Заряд!свободный}, то~они начнут перемещаться по~проводнику. Проводник в~результате поляризуется\index{Поляризация} и~сам создает поле\index{Поле}. Поток зарядов в~нем прекратится, когда внутреннее поле\index{Поле!внутреннее} скомпенсирует внешнее. Таким образом, проводником будем называть вещество, в~котором всегда достаточно заряда, чтобы скомпенсировать внутри себя любое внешнее поле.