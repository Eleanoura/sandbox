\section{Потенциалы. Силовые линии и эквипотенциальные поверхности}

	Покажем, что электрическая сила консервативна. В силу принципа суперпозиции
		$$\vec{E}=\sum_i \vec{E}_i,$$
	откуда
		$$A=\sum_i A_i.$$
	Поле электрического заряда центрально симметричное, следовательно работа электрических сил по замкнутому контуру равна нулю:
		$$\oint \vec{F}\cdot\d\vec{l}=q\oint \vec{E}\cdot\d\vec{l}=0.$$
	Если есть консервативная сила, то есть и потенциальная энергия. Например, силе $\vec{F}_{\text{грав}}=G\dfrac{m_1m_2\vec{r}}{r^3}$ соответствует потенциальная энергия $E=-G\dfrac{m_1m_2}{r}$. Рассуждая аналогично, определим \i{потенциальную энергию электрического поля, порождаемого зарядом}:
		$$E_{\text{п}}=+k\frac{q_1q_2}{r}.$$
	В соответствии с~принципом суперпозиции
		$$E_{\text{п}}=\sum_i E_i=\sum_i k\frac{qq_i}{r_i}=q\sum_i k\frac{q_i}{r_i}.$$
	Скалярную величину $\varphi=\dfrac{E_{\text{п}}}{q}$ назовем \i{электрическим потенциалом точки}.
		$$\dim{\varphi}=\frac{\text{Дж}}{\text{Кл}}=\text{В (вольт)}.$$
	Для потенциала также выполняется принцип суперпозиции:
		$$\varphi=\sum_i \varphi_i=\sum_i k\frac{q_i}{r_i}.$$
	Потенциал действует на заряды и создается зарядами. Знак потенциала соответствует знаку зарада, его породившего. \par
	Пусть заряд $q$ передвигается в~электрическом поле из точки 1 в 2. Тогда работа электрической силы запишется как
		$$A=E_{\text{п}_{\text{1}}}-E_{\text{п}_{\text{2}}}=q\varphi_1-q\varphi_2=q(\varphi_1-\varphi_2).$$
	Назовем величину $U=\varphi_1-\varphi_2$ \i{напряжением} и запишем работу $A$ в виде 
		$$A=qU.$$
	Заряд $q$ называется \i{пробным}, если он достаточно мал, чтобы в~условии данной задачи не~менять распределение и~картину поля от~всех остальных зарядов.\par
	Для визуального представления полей использутся силовые линии~--- воображаемые линии, в~каждой точке сонаправленные с~вектором напряженности электрического поля в~этой точке. Густота~--- величина $\Gamma=\dfrac{N}{S}$~--- это отношение числа $N$ силовых линий, проходящих через единицу площади $S$, к~$S$; иначе говоря, густота~--- это <<плотность>> силовых линий.