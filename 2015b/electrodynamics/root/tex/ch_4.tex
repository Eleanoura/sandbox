\section{Уравнения электростатики}

	Закон Кулона и~принцип суперпозиции позволяет подсчитать поле и~силу в~любой точке пространства. Альтернативный способ сделать это --- уравнения Максвелла. \par
	Вспомним, что электрическое поле центрально симметрично и~консервативно:
		$$A=\oint \vec{F} \dvec\vec{l}=q\oint E_l\d{l}=0,$$
	тогда запишем равенство, называемое \i{вторым уравнением Максвелла}:
		\begin{equation}
			\oint E_l\d{l} =0.
		\end{equation}
	\i{\b{Циркуляция электростатического поля равна нулю, что отражает консервативность этого поля}}. Это значит, что \, замкнутых силовых линий нет.\par
	Докажем формулу Гаусса-Остроградского.\par
	Из каждой замкнутой поверхности выходит число силовых линий, пропорциональное суммарному заряду:
		$$N\prop q_{\Sigma}.$$
	Тогда
		$$\Gamma=\frac{dN}{dS}\prop E, \quad E\, dS\prop dN.$$
	Величину 
		$$\dvec{\Phi_E}=\vec{E}\d S$$
	назовем потоком электрического поля. Для произвольной поверхности поток электрического поля сквозь нее равен
		\begin{equation}
			\Phi_E=\int\limits_S E_n\d{S}.
		\end{equation}
	Таким образом мы~определили понятие, аналогичное интуитивному понятию густоты. Теперь
		\begin{equation}\label{eq:prop1}
			N\prop\oint E\d{S}\prop q_{\Sigma}.
		\end{equation}
	\i{\b{Поток электрического поля через замкнутую поверхность пропорционален суммарному заряду внутри поверхности.}} Поток считается положительным, если поле идет наружу.\par
	Запишем~(\ref{eq:prop1}) с коэффициентом пропорциональности:
		\begin{equation}
			\oint E_n\d{S}=\frac{q}{\varepsilon_0},
		\end{equation}
	где $\varepsilon_0$ -- величина, называемая диэлектрической проницаемостию вакуума. Пропорцию~(\ref{eq:prop1}) можно записать в~таком виде, поскольку $E\prop\dfrac{1}{R^2}$, $S\prop R^2$, а~тогда $ES=\c\prop q$.\par
	Рассмотрим теперь простую сферическую поверхность с зарядом внутри. Тогда
		$$\Phi_E=4\pi R^2 E=\frac{q}{\varepsilon_0},$$
	откуда
		$$E=\frac{1}{4\pi\varepsilon_0}\frac{q}{R^2}.$$
	Но $E=k\dfrac{q}{R^2}$, значит,
		\begin{equation}
			k=\frac{1}{4\pi\varepsilon_0}.
		\end{equation}
	Имеем $\varepsilon_0=8,85\times 10^{-12} \dfrac{\text{Кл}^2}{\text{Н}\cdot\text{м}}$.