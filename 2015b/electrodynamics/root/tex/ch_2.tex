\section{Электрическое поле. Напряженность \,\, электрического поля.}

	Сила $\vec{F}$, действующая на заряд $q$, всегда пропорциональна его величине, поэтому~(\ref{2}) можно записать в~виде
		\begin{equation}
			\vec{F} = q\vec{E},
		\end{equation}
	где вектор $\vec{E}$ называют \i{вектором напряженности электрического поля} \index{Вектор!напряженности электрического поля}. Это аналог формулы $\vec{F}=m\vec{g}$. $\vec{E}$ и $\vec{g}$ являются характеристиками данной точки пространства.
		$$\dim{E}=\frac{\text{Н}}{\text{Кл}},$$
		$$\vec{F}=q\sum_i \frac{q_i\vec{r}_i}{r_i^3}=q\sum_i \vec{E}_i=q\vec{E}.$$
	Поле \index{Поле!электрическое} в~данной точке есть суперпозиция \index{Принцип!суперпозиции} полей, порождаемых всеми зарядами в~системе.
		$$\vec{E}=\sum_i \vec{E}_i,$$
		$$\vec{E}_i=k\frac{q_i\vec{r}_i}{r_i^3}, \quad E_i = k\frac{q_i}{r_i^2}.$$
	Электрическое поле создается зарядами и~действует на~заряды. Заряды не~действуют друг на~друга и~взаимодействуют посредством полей, которые создают.